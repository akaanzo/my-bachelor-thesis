\cleardoublepage
\chapter{Teoria dei fili ideali}\label{chapter:teoria_fili}
Nella trattazione analitica (derivante da \cite{siboni:funi}), quando si parla di filo (o fune), si intende un solido la cui lunghezza longitudinale sia molto maggiore della lunghezza caratteristica della sezione (diametro o diagonale, in base alla forma), con scarsa resistenza a flessione e torsione e che sopporti solo sforzi di trazione.

In questo modo è possibile descrivere il filo mediante una curva di parametrizzazione

\begin{equation*}
	P=P(\lambda) \quad \lambda\in[\lambda_1, \lambda_2]
\end{equation*}
La curva deve risultare sufficientemente regolare e di classe almeno $C^2$.
Per definizione, i punti $P(\lambda_1)$ e $P(\lambda_2)$ vengono detti \emph{estremi del filo}.

La teoria qui riportata fa riferimento ai cosiddetti \emph{fili ideali}, definiti da due caratteristiche principali:
\begin{enumerate}
	\item configurazioni ammissibili;
	\item reazioni vincolari.
\end{enumerate}

\section*{Configurazioni ammissibili}
Un filo ideale può assumere qualsiasi configurazione descritta da una curva $C^2$ regolare (o biregolare) di lunghezza $L$ assegnata. Questo significa:
\begin{itemize}
	\item la curvatura $1/\,\rho$ e la torsione $1/\,\sigma$ possono assumere valori arbitrari come si deduce dalle formule di Frenet -- Serret
	\begin{equation*}
		\dfrac{d\hat{\tau}}{ds} = \dfrac{1}{\rho}\,\hat{n} \hspace*{2cm} \dfrac{d\hat{b}}{ds} = - \dfrac{1}{\sigma}\,\hat{n}
	\end{equation*}
	ove $s$ è l'ascissa curvilinea e $\hat{\tau}$, $\hat{n}$, $\hat{b} = \hat{\tau} \wedge \hat{n}$ sono rispettivamente i versori tangente, normale e binormale alla curva nella posizione $P(s)$.

	L'arbitrarietà di questi valori rende il filo ideale perfettamente deformabile sia a flessione che a torsione, cosa che non accade in una fune reale: infatti, curvature molto grandi, genererebbero un momento flettente elevato nelle fibre interne del materiale che causerebbe la rottura dell'elemento. In particolare, curvatura e momento flettente sono legati dalla seguente legge caratteristica
	\[
	M = E\,I\,\chi
	\]
	dove $E$ è il modulo elastico del materiale, $I$ è il momento di inerzia della sezione e $\chi$ è la curvatura.

	Si assume, quindi, rigidezza flessionale nulla ($E\,I\simeq 0$).
	\item la lunghezza $L$ del filo è costante; le configurazioni $P(\lambda)$ devono soddisfare la condizione di \emph{inestensibilità}
	\[
	\int_{\lambda_1}^{\lambda_2} \left| \dfrac{dP}{d\lambda}(\lambda) \right|\,d\lambda = \int_{\lambda_1}^{\lambda_2} [\dot{P}(\lambda)^2]^{1/2}\, d\lambda = L = cost
	\]
	la cui impone che il filo, anche se sottoposto a sollecitazioni, non subisce variazioni di lunghezza. Anche in questo caso si tratta di una approssimazione della realtà poiché una fune reale, soggetta a sollecitazioni più o meno grandi, si allunga secondo il legame costitutivo del materiale.

	La condizione di inestensibilità postula una rigidezza assiale infinita ($E\,A \to \infty$).
\end{itemize}

\section*{Reazioni vincolari}
Il filo ideale sottoposto alle sollecitazioni reagisce internamente con delle reazioni vincolari interne.

Considerando un intervallo $\lambda\in[\lambda_1, \lambda_2]$ si postula che le reazioni vincolari interne ad un tratto di filo (ad esempio $P(\lambda) - P(\lambda_2)$) si riducano a una unica reazione vincolare applicata in $P(\lambda)$. Questa forza, indicata con $\overline{T}(\lambda)$, prende il nome di \emph{tensione del filo} e risulta essere tangente alla curva in $\lambda$ (vedi figura~\ref{fig:tensione_filo}).

\begin{figure}[ht]
		\centering
		
		\subfloat[\emph{Tensione in un tratto di filo}\label{fig:tensione_filo}]{
		\begin{tikzpicture}[scale=.8]
		
			
			
			\draw[line width=1pt] (0,0)node [left] {$P(\lambda_1)$} .. controls ++(1.5,3) and ++(.1,-2) .. (10,4) node [right] {$P(\lambda_2)$}
			node[sloped, pos=.3, anchor=south west,
			minimum height=(10.5)*0.3cm,minimum width=(10.5)*.3cm](p){};
			
			\path (p.south west)%
		%	edge[-stealth',blue] node[left] {$\vec{ n}$} (p.north west)
			edge[-stealth',blue, thick] node[pos=1, above] {$\overline{T}(\lambda)$} (p.south east);
			
			\node at (p.south west) [below] {$P(\lambda)$};
			
			
			
		
		
		
		
 						\end{tikzpicture}}
 		
 		\subfloat[\emph{Estensione del postulato a una fune ideale}\label{fig:tensione_fune_ideale}]{
 		
 		
 		
 		
 \begin{tikzpicture}[scale=.9]
  
 \draw plot  [smooth, tension=1] coordinates {(0,0) (1,.3) (3,.6)};
 \begin{scope}[shift={(0,-1)}]
 \draw plot  [smooth, tension=1] coordinates {(0,0) (1,.3) (3,.6)};
 \end{scope}
 
 \draw (0,0) to [out=-20, in=40] (0,-1);
 \draw (0,0) to [out=-145, in=130] (0,-1);
 
 \draw (3,.6) to [out=-20, in=40] (3,-.4);
 \draw [dashed] (3,.6) to [out=-145, in=130] (3,-.4);
 
 \draw [-stealth', blue] (3,.1) --([turn] 8:1) node [above] {\small$\overline{T}(\lambda)$};
 \draw [-stealth, red] (0,-.5) --([turn] 290:1) node [below] {\small$\overline{T}(\lambda_1)$};
 
 \node at (0,0) [left] {$P(\lambda_1)$};
 \node at (3,.6) [above] {$P(\lambda)$};
 
 \begin{scope}[shift={(9,0.35)}, yscale=1, xscale=-1]
 \draw plot  [smooth, tension=1] coordinates {(0,0) (1,.3) (3,.6)};
 \begin{scope}[shift={(0,-1)}]
 \draw plot  [smooth, tension=1] coordinates {(0,0) (1,.3) (3,.6)};
 \draw [-stealth, red] (0,.5) --([turn] 100:1) node [below] {\small$\overline{T}(\lambda_2)$};
 \end{scope}
 
 \draw [dashed](0,0) to [out=-20, in=40] (0,-1);
 \draw [] (0,0) to [out=-145, in=130] (0,-1);
 
 \draw (3,.6) to [out=-20, in=40] (3,-.4);
 \draw [] (3,.6) to [out=-145, in=130] (3,-.4);
 
 \draw [-stealth', blue] (3,.1) --([turn] -8:1) node [above] {\small$-\overline{T}(\lambda)$};

 
 \node at (0,0) [right] {$P(\lambda_2)$};
 \node at (3,.6) [above] {$P(\lambda)$};
 \end{scope}
 \end{tikzpicture}


}
		
		
		
		
		
		
		
		
		
		\caption{Tensione in un tratto di filo}
		\label{fig:tensione_fune}
		
	\end{figure}
	
	
	


Estendendo quanto appena visto a una fune ideale di sezione non puntuale e tagliando in corrispondenza di $P(\lambda)$, per il \emph{principio di azione e reazione}, il tratto residuo genera una tensione uguale ed opposta, come visibile in figura~\ref{fig:tensione_fune_ideale}.

Poiché il filo ideale è un elemento puntuale, il momento delle reazioni vincolari interne agenti su ogni sezione -- rispetto al baricentro -- risulta nullo. Ne consegue la condizione di \emph{perfetta flessibilità}, descritta poc'anzi, che è rafforzata dalla richiesta di  resistenza nulla a sforzi di taglio da parte del filo dovuta al fatto che la tensione è tangente al supporto del filo.

Nella realtà, avendo -- il filo -- un diametro non infinitesimo, il momento generato dalle reazioni vincolari interne è, generalmente, diverso da zero.

Infine, l'arbitrarietà del valore della tensione (che si impone di sola trazione) comporterebbe un carico di rottura del filo infinito, condizione che non è assolutamente applicabile in caso di fune reale.

\subsection*{Osservazione}
Mentre la tensione $\overline{T}(\lambda)$ è definita come una reazione vincolare interna, agli estremi, i termini $\overline{T}(\lambda_1)$ e $\overline{T}(\lambda_2)$ sono descritte come forze attive che interagiscono con l'esterno come, per esempio, un vincolo; in questo caso le tensioni $\overline{T}(\lambda_1)$ e $\overline{T}(\lambda_2)$ si identificano con i \emph{cimenti dinamici}.

Definito il triedro di Frenet in un punto $P(\lambda)$ della curva, si può riscrivere la tensione come
\[
	\overline{T}(\lambda) = T(\lambda)\,\hat{\tau}(\lambda)
\]
ove
\[
T(\lambda) = \left|\overline{T}(\lambda)\right|\geq 0 \quad \lambda\in[\lambda_1, \lambda_2]
\]

\section{Sollecitazioni esterne}
Esistono due diverse tipologie di sollecitazioni che possono essere applicate al filo:
\begin{itemize}
	\item concentrate: sono applicate in un punto singolo del filo e sono descritte da un vettore;
	\item distribuite: sono applicate lungo un tratto (oppure tutto) di filo e sono descritte da una densità di forza per unità di lunghezza $\overline{f}(s)$. Preso un tratto infinitesimo di filo di lunghezza $ds$, la forza che agisce sul tratto considerato vale
	\[
		d\overline{F}(s) = \overline{f}(s)\,ds \quad s\in[0,L]
	\]
	ove $s$ è la \emph{ascissa curvilinea} che descrive il filo di intervallo $[0,L]$.
	In generale, le sollecitazioni sono funzione di posizione e velocità oltre che del tempo
	\[
	\overline{F} = \overline{F}(t, P, \dot{P})
	\]
\end{itemize}

\section{Condizione di equilibrio}
Per un sistema costituito da $N$ punti materiali ${P_1, \dots, P_N}$ soggetto a sollecitazioni $\overline{F}_i (t, P_i, \dot{P}_i),~ i=1,\dots,N$ la configurazione $P_0$ è di equilibrio \emph{se e solo se} la quiete in $P_0$, definita come
\[
P(t) = P_0\quad \forall t\in\mathbb{R}
\]
è un \emph{moto naturale} del sistema, cioè è soluzione delle equazioni del moto.

Il postulato delle reazione vincolari in termini generali è descritto dalla seguente equazione
\[
m_i\,\ddot{P}_i = \overline{F}_i(t,P_i(t), \dot{P}_i(t)) + \overline{\Phi}_i\qquad i=1,\dots,N
\]
Applicandolo alla configurazione $P_0$, indicato precedentemente come in quiete, si ha
\begin{equation}
	\label{eq:equilibrio_p0}
	0 = \overline{F}_i (t, P_0, 0) + \overline{\Phi}_i\qquad i = 1,\dots, N\quad\forall t\in\mathbb{R}
\end{equation}
dove le reazioni vincolari $\overline{\Phi}_i$ devono essere quelle effettivamente esplicabili dai vincoli.

L'equazione \eqref{eq:equilibrio_p0} equivale alla \emph{prima equazione cardinale della statica}; definito un polo $O\in\mathbb{E}^3$, la \emph{seconda equazione cardinale della statica}, che deriva esplicitamente dalla prima, risulta
\begin{equation*}
	0 = (P_i - O)\wedge\overline{F}_i(t,P_0, 0) + (P_i - O)\wedge\overline{\Phi}_i\qquad i=1,\dots,N\quad\forall t\in\mathbb{R}
\end{equation*}
Allora si può dire che, \emph{condizione necessaria e sufficiente} perché $P_0$ sia un equilibrio del sistema è che siano valide le equazioni cardinali della statica, in conformità con le reazioni vincolari esplicate dai vincoli.

Per quanto riguarda un corpo continuo come un filo, si dice che:
\begin{quotation}
	una configurazione del filo è in equilibrio se e solo se sono soddisfate, per mezzo di reazioni vincolari esplicabili dai vincoli, le equazioni cardinali della statica \textbf{per ogni tratto di fune}.\footnote{citato da \cite{siboni:funi}}
\end{quotation}

\section{Equazioni cardinali della statica per un elemento di filo}




\begin{figure}
	\centering
	
	\begin{tikzpicture}[scale=.6]
	
	\draw [thick] (0,0) .. controls (5,-.5) .. (10,.5);
	
	
	
	\draw [-latex, thick] (0,-.25) -- ++([turn] 75:-1.5) node [above left] {$-\overline{T}(s)$};
	
	\draw [-latex, thick] (10,.25) -- ([turn] 12:2) node [above right] {$\overline{T}(s + \delta s)$};
	
	
	
	
	\draw [thick](0,-.5) node [below] {$P(s)$} .. controls (5,-1)..(10,0) node [below right] {$P(s+\delta s)$};
	
	\draw[thick] (0,0) .. controls (.2,-.18) and (.2,-.36) .. (0,-.5);
	\draw[thick] (0,0) .. controls (-.2,-.18) and (-.2,-.36) .. (0,-.5);
	
	\draw [thick](10,0.5) .. controls (10.1,.3) and (10.1,.1) .. (10,0);
	\draw [thick, dashed](10,0.5) .. controls (9.9,.3) and (9.9,.1) .. (10,0);
	
	
	
	\begin{scope}[shift={(0,1.3)}]
	\draw [thick]  (.15,0) .. controls (5,-.5) .. (9.7,.49);
	\draw [forcedist=.75cm]  (0,0) .. controls (5,-.5) ..(10,.5);
	\node  at (5,0.2) {$\overline{f}(\xi)$};
	\end{scope}
	
	
	\end{tikzpicture}
	
	\caption{Forze esterne agenti su un tratto di filo}
	\label{fig:forze_filo}
	
\end{figure}




\subsection{Prima equazione cardinale della statica}

Si consideri un tratto di filo compreso fra $P(s)$ e $P(s + \delta s)$ (figura~\ref{fig:forze_filo}); per l'equilibrio è necessario che il vettore risultante di tutte le sollecitazioni agenti sul tratto di filo sia nullo
\[
\overline{T}(s+\delta s) - \overline{T}(s) + \int_s^{s+\delta s} \overline{f}(\xi)\,d\xi = 0
\]
Proiettando la densità di forza lungo un generico versore $\hat{e}_i$ e applicando il teorema del valor medio, essendo $\overline{f}(\xi)$ una funzione continua nell'intervallo $\xi\in [s, s +\delta s]$, si può scrivere
\[
\hat{e}_i \cdot \int_s^{s+\delta s} \overline{f}(\xi)\,d\xi = \int_s^{s+\delta s} \hat{e}_i\cdot \overline{f}(\xi)\,d\xi = \delta s\,\hat{e}_i\cdot\overline{f}(s+ \theta_i\,\delta s)
\]
con $\theta_i\in (0,1)$ opportuno. Generalizzando per le tre direzioni $1, 2, 3$ e sostituendo nell'equazione si ottiene
\begin{equation*}
\overline{T}(s+\delta s) - \overline{T}(s) + \delta s\,\sum_{i=1}^3 \hat{e}_i\cdot \overline{f}(s+\theta_i\,\delta s)\,\hat{e}_i = 0
\end{equation*}

A questo punto, dividendo entrambi i membri per $\delta s$ e calcolando il seguente limite
\[
\lim_{\delta s\to 0} \left[ \dfrac{\overline{T}(s+\delta s)-\overline{T}(s)}{\delta s} + \sum_{i=1}^3 \hat{e}_i\cdot \overline{f}(s+\theta_i\,\delta s)\,\hat{e}_i \right] = 0
\]
Poiché $\overline{f}$ è una funzione continua, il limite esiste e vale
\[
\lim_{\delta s \to 0} \overline{f}(s+\theta_i\,\delta s) = \lim_{\delta s \to 0} \overline{f}(s+\delta s) = \overline{f}(s)
\]
da cui
\[
\lim_{\delta s \to 0} \sum_{i=1}^3 \hat{e}_i\cdot \overline{f}(s+\theta_i\,\delta s)\,\hat{e}_i = \sum_{i=1}^3 \hat{e}_i\cdot \overline{f}(s)\,\hat{e}_i = \overline{f}(s)
\]
La definizione di limite, inoltre, assicura che
\[
\lim_{\delta s \to 0} \dfrac{\overline{T}(s+\delta s)-\overline{T}(s)}{\delta s} = \dfrac{d\overline{T}}{ds}(s)
\]
In definitiva, la \textbf{prima equzione cardinale della statica} si riscrive nella seguente forma
\begin{equation}
\label{eq:prima_eq_cardinale_statica}
\dfrac{d\overline{T}}{ds}(s) + \overline{f}(s) = 0\qquad \forall s \in (s_1, s_2)
\end{equation}
che risulta una equazione differenziale nella coordinata $s$.

\subsection{Seconda equazione cardinale della statica}
Scegliendo come polo per il calcolo del momento angolare il punto $P(s)$, la seconda equazione della statica si può scrivere come
\[
[P(s+\delta s) - P(s)] \wedge \overline{T}(s+\delta s) + \int_s^{s+\delta s} [P(\xi) - P(s)]\wedge \overline{f}(\xi)\,d\xi = 0
\]
Come fatto in precedenza, si proietta l'equazione lungo i versori degli assi coordinati e si applica il teorema del valor medio
\begin{align*}
&\hat{e}_i \cdot \int_s^{s+\delta s} [P(\xi) - P(s)]\wedge \overline{f}(\xi)\,d\xi =\\ =&\int_s^{s+\delta s} \hat{e}_i \cdot [P(\xi) - P(s)]\wedge \overline{f}(\xi)\,d\xi =\\=& \delta s\,\hat{e}_i \cdot [P(s+\zeta_i\,\delta s) - P(s)] \wedge \overline{f}(s+\zeta_i\,\delta s)
\end{align*}
ove $\zeta_i \in (0,1)$ è un opportuno scalare. Sostituendo e sommando vettorialmente lungo le tre direzioni si ottiene
\begin{align*}
&[P(s+\delta s) - P(s)] \wedge \overline{T}(s+\delta s) +\\+& \delta s\,\sum_{i=1}^3 \hat{e}_i \cdot [P(s + \zeta_i\,\delta s) - P(s)]\wedge \overline{f}(s + \zeta_i\,\delta s)\,\hat{e}_i = 0
\end{align*}
Si divide, quindi, per la lunghezza $\delta s$ e si calcola il limite per $\delta s$ tendente a zero
\begin{align*}
&\lim_{\delta s\to 0} \left[ \dfrac{P(s+\delta s) - P(s)}{\delta s} \wedge \overline{T}(s + \delta s)\right] +\\
+&\lim_{\delta s\to 0} \,\sum_{i=1}^3 \hat{e}_i \cdot [P(s+\zeta_i\,\delta s) - P(s)]\wedge \overline{f}(s+\zeta_i\,\delta s)\,\hat{e}_i = 0
\end{align*}
Il secondo termine si annulla, poiché
\begin{align*}
&\lim_{\delta s\to 0} \hat{e}_i \cdot [P(s + \zeta_i\,\delta s) - P(s)] \wedge \overline{f}(s+\zeta_i\,\delta s) =\\
=& \lim_{\delta s \to 0} \hat{e}_i \cdot [P(s + \delta s) - P(s)] \wedge \overline{f}(s+\delta s) =\\
=& \hat{e}_i \cdot [P(s) - P(s)]\wedge \overline{f}(s) = 0
\end{align*}
mentre il primo termine
\begin{align*}
&\lim_{\delta s \to 0} \dfrac{P(s+\delta s) - P(s)}{\delta s} = \dfrac{dP}{ds}(s)\\[1.5ex]
&\lim_{\delta s \to 0} \overline{T}(s+\delta s) = \overline{T}(s)
\end{align*}
Allora, \textbf{seconda equazione cardinale della statica} riscritta in forma locale è
\begin{equation}
\label{eq:seconda_eq_cardinale_statica}
\dfrac{dP}{ds}(s) \wedge \overline{T}(s) = 0\qquad \forall s\in (s_1, s_2)
\end{equation}

\subsubsection*{Osservazione}
Ricordando la definizione del versore tangente
\[
\hat{\tau}(s) = \dfrac{dP}{ds}(s)
\]
la \eqref{eq:seconda_eq_cardinale_statica} assicura, grazie all'annullarsi del prodotto vettoriale, che la tensione $\overline{T}(s)$ sia tangente al filo
\[
\overline{T}(s) = T(s)\,\dfrac{dP}{ds}(s) = T(s)\,\hat{\tau}(s)
\]

Le sole equazioni cardinali della statica \eqref{eq:prima_eq_cardinale_statica} e \eqref{eq:seconda_eq_cardinale_statica} descrivono una \emph{condizione necessaria ma non sufficiente} per l'equilibrio del filo.

La sufficienza della suddetta condizione viene garantita richiedendo che le tensioni si mantengano di segno non negativo; perciò
\[
T(s)\geq 0\qquad \forall s\in(s_1, s_2)
\]
dove le tensioni sono dirette in modo concorde all'ascissa curvilinea $s$.

In definitiva, nel caso di filo soggetto a sollecitazioni distribuite di densità $\overline{f}(s)$ continua nell'intervallo $(s_1, s_2)$, \emph{condizione necessaria e sufficiente per l'equilibrio del filo} è che si soddisfino le seguenti relazioni
\begin{equation}
	\label{eq:cns_equilibrio}
	\begin{cases}
		\dfrac{d\overline{T}}{ds}(s) + \overline{f}(s) = 0\\
		\overline{T}(s) = T(s)\,\dfrac{dP}{ds}(s)\\
		T(s) \geq 0
	\end{cases}
	\quad \forall s \in(s_1, s_2)
\end{equation}

\section{Equazioni intrinseche dell'equilibrio}
Nelle ipotesi di curva biregolare  e di modulo della tensione  $T(s)$ strettamente positivo,  si può ridurre il sistema \eqref{eq:cns_equilibrio} a una unica equazione
\begin{equation}
	\label{eq:equzione_indefinita_equilibrio}
	\dfrac{d}{ds}(T\,\hat{\tau}) + \overline{f}(s) = 0
\end{equation}
detta \emph{equazione indefinita di equilibrio}. Sviluppando la derivata del prodotto presente nel primo termine si ottiene
\[
	\dfrac{dT}{ds}\,\hat{\tau} + T\,\dfrac{d\hat{\tau}}{ds} + \overline{f}(s) = 0
\]
e ricordando la definizione di curvatura che lega la derivata del versore tangente al versore normale
\[
	\dfrac{dT}{ds}\,\hat{\tau} + T\,\dfrac{1}{\rho}\,\hat{n} + \overline{f}(s) = 0
\]
Proiettando l'equazione appena ricavata lungo i versori che formano il triedro di Frenet si ottengono le \emph{equazioni intrinseche dell'equilibrio} per il tratto di filo considerato
\begin{equation}
	\begin{cases}
		\dfrac{dT}{ds} + \hat{\tau}\cdot \overline{f}(s)=0\\[2ex]
		\dfrac{T}{\rho} + \hat{n}\cdot \overline{f}(s)=0\\[2ex]
		\hat{b}\cdot \overline{f}(s)=0
	\end{cases}
\end{equation}

\subsection*{Osservazione}
La densità di forza distribuita $\overline{f}(s)$, finora scritta come funzione della sola variabile s, è in realtà una funzione nota dipendente non solo dall'ascissa curvilinea $s$ ma anche dalla parametrizzazione $P(s)$ e del versore tangente $\hat{\tau}(s)$; Si considera quindi
\[
\overline{f} = \overline{f}	(s, P(s), \hat{\tau}(s))
\]

Nella trattazione seguente si considererà implicita la dipendenza da $s$ delle funzioni tensione, versore tangente e densità di sollecitazione.

\section{Equazioni di equilibrio ridotte a forma normale}
Sia $T(s)>0\quad\forall s \in[s_1, s_2]$ e derivabile nello stesso intervallo; l'equazione indefinita di equilibrio \eqref{eq:equzione_indefinita_equilibrio} può essere riscritta come
\begin{equation}
	\label{eq:equazione_indefinita_equilibrio_1}
	\dfrac{dT}{ds}\,\hat{\tau} + T\,\dfrac{d\hat{\tau}}{ds}+ \overline{f} = 0
\end{equation}

Moltiplicando scalarmente i termini per il versore tangente si ha
\[
	\dfrac{dT}{ds}\,\hat{\tau}^2  + \overline{f}\cdot\hat{\tau} = 0
\]
e, per ottenere la variazione di $T$ in $s$ è sufficiente dividere per $\hat{\tau}^2$ e portare il secondo termine a secondo membro
\[
\dfrac{dT}{ds} = -\dfrac{\overline{f}\cdot\hat{\tau}}{\hat{\tau}^2}	
\]
Sostituendo quanto appena ricavato nella \eqref{eq:equazione_indefinita_equilibrio_1} si ottiene
\[
	-\dfrac{\overline{f}\cdot\hat{\tau}}{\hat{\tau}^2}\,\hat{\tau} + T\,\dfrac{d\hat{\tau}}{ds}+ \overline{f} = 0
\]
e riorganizzando il tutto
\[
\dfrac{d\hat{\tau}}{ds} = \dfrac{1}{T} \left(-\overline{f} + \dfrac{\overline{f}\cdot\hat{\tau}}{\hat{\tau}^2}\,\hat{\tau}\right)
\]

Ricordando, inoltre, la definizione di $\hat{\tau}$ si ottiene un sistema di 7 equazioni in 7 incognite ($P, T, \hat{\tau}$) funzioni di $s$
\begin{equation}
	\begin{aligned}[c]
		\begin{cases}
			\dfrac{dP}{ds} = \hat{\tau}\\[2ex]
			\dfrac{d\hat{\tau}}{ds} = \dfrac{1}{T} \left(-\overline{f} + \dfrac{\overline{f}\cdot\hat{\tau}}{\hat{\tau}^2}\,\hat{\tau}\right)\\[2ex]
			\dfrac{dT}{ds} = -\dfrac{\overline{f}\cdot\hat{\tau}}{\hat{\tau}^2}
		\end{cases}
	\end{aligned}
	\qquad 
	\begin{aligned}
		&\forall s\in[s_1, s_2]\\
		& P\in\mathbb{R}^3\\
		&\hat{\tau} \neq 0
	\end{aligned}
\end{equation}

La definizione di \emph{versore} impone che
\[
\left| \hat{\tau}(s)\right| = 1 \qquad \forall s\in[s_1, s_2]
\]
il che assicura, in un punto intermedio $s_1<s_0<s_2$
\[
	\left| \hat{\tau}(s_0)\right| = 1 
\]
e perciò
\[
\hat{\tau}(s_0)= \hat{\tau}_0,\qquad \left|\hat{\tau}_0\right|^2 = 1
\]
Applicando, così, le \textbf{condizioni iniziali} si perviene al seguente problema di Cauchy
\begin{equation}
	\label{eq:problema_cauchy}
	\begin{cases}
		\dfrac{dP}{ds} = \hat{\tau}\\[1.5ex]
		\dfrac{d\hat{\tau}}{ds} = \dfrac{1}{T} \left(-\overline{f} + \dfrac{\overline{f}\cdot\hat{\tau}}{\hat{\tau}^2}\,\hat{\tau}\right)\\[1.5ex]
		\dfrac{dT}{ds} = -\dfrac{\overline{f}\cdot\hat{\tau}}{\hat{\tau}^2}\\[1.5ex]
		P(s_0) = P_0\\
		\hat{\tau}(s_0)= \hat{\tau}_0\\
		T(s_0)=T_0
	\end{cases}
\end{equation}
ove la funzione nota $\overline{f}\in C^1$ in $s$, $P$, $\hat{\tau}$ e il dominio di definizione è
\[
\Omega = \left\{(s, P, \vtau, T)\in \mathbb{R}\times\mathbb{R}^3\times \mathbb{R}^3\texttt{\textbackslash}\left\{0\right\}\times \mathbb{R}^+\right\}
\]

Il teorema di esistenza e unicità applicato al problema di Cauchy \eqref{eq:problema_cauchy} assicura che la soluzione massimale esiste ed è unica.
La soluzione ricercata è la curva funicolare (una e una sola) per cui il punto del filo $s_0$, di versore tangente $\vtau_0$, si colloca nella posizione $P_0$ nello spazio ed è sottoposto alla tensione $T_0$.

\subsection{Condizioni al contorno}\label{section:condizioni_contorno}
Il problema di Cauchy \eqref{eq:problema_cauchy} è stato risolto imponendo dei valori iniziali. Tuttavia, nella realtà, ci si trova di fronte a un problema di carattere diverso.

Si consideri un filo di estremi $A$ e $B$, con lunghezza $L$ nota. La parametrizzazione della curva è definita nell'intervallo $[s_1, s_2] = [0, L]$, mentre gli estremi del filo sono definiti come
\[
P(s_1)	= A \qquad P(s_2)=B
\]

Applicando al problema i valori noti delle posizioni agli estremi del filo, ci si riconduce non più a un problema ai valori iniziali -- per cui si impongono i valori nella sola coordinata $s_0$ -- ma a un \textbf{problema a valori al contorno}.

L'uso delle condizioni al contorno non assicura l'esistenza e l'unicità della soluzione. Infatti, il teorema di esistenza e unicità fa riferimento al solo problema ai valori iniziali. Il sistema può essere comunque risolto con metodi numerici nonostante la soluzione possa esistere (una o più di una) o meno.

In conclusione si può osservare come, assegnando le posizioni degli estremi, si possano determinare le tensioni che gli essi generano
\[
s_1: T(s_1)\,\vtau(s_1)\qquad s_2: T(s_2)\,\vtau(s_2)
\]
e quindi le reazioni vincolari necessarie per l'equilibrio
\[
\overline{T}_A = -	T(s_1)\,\vtau(s_1)\qquad \overline{T}_B = -T(s_2)\,\vtau(s_2)
\]

\subsection{Calcolo delle tensioni}
La tensione $T(s)$ può essere ricavata dall'equazione indefinita di equilibrio 
\[
\dfrac{d}{ds}\,\left[T(s)\,P'(s)\right] + \overline{f}(s, P(s), P'(s)) = 0
\] 
integrando tra l'estremo inferiore $s_1$ e un generico estremo $s \in [s_1, s_2]$
\[
T(s)\,P'(s) - T(s_1)\,P'(s_1) + \int_{s_1}^s \overline{f}(\xi, P(\xi), P'(\xi))\,d\xi = 0
\]
da cui si ricava facilmente
\begin{align*}
\overline{T}	(s) = T(s)\,P'(s) =& T(s_1)\,P'(s_1) - \int_{s_1}^s \overline{f}(\xi, P(\xi), P'(\xi))\,d\xi\\
=& - \overline{T}_A - \int_{s_1}^s \overline{f}(\xi, P(\xi), P'(\xi))\,d\xi 
\end{align*}



\section{Campo di forze continue e parallele}\label{section:forze_continue_parallele}
In presenza di un campo di forze di direzione costante, la densità di forza si può scrivere nella forma
\[
\overline{f}(s)	= f(s)\,\hat{u}
\]
dove $\hat{u}$ è un versore costante e $f(s)$ è una funzione continua nel suo intervallo di definizione $[s_1, s_2]$.

L'equazione indefinita di equilibrio \eqref{eq:equzione_indefinita_equilibrio} diventa
\[
\dfrac{d}{ds}	(T\,\vtau) + f(s)\,\hat{u} = 0
\]

Allora, presa una coordinata $s_0 \in[s_1, s_2]$ 
\[
\dfrac{d}{ds}	\left[ T\,\vtau + \int_{s_0}^s f(\xi)\,d\xi\,\hat{u}\right] = 0\qquad \forall s\in[s_1, s_2]
\]
risulta un \emph{integrale primo}; questo implica che il termine all'interno della parantesi sia costante e valga
\[
\overline{R}_0 = T(s_0)	\,\vtau(s_0) = cost.
\]
 Si può, quindi, ricavare la tensione invertendo la formula
 \[
T(s)\,\vtau(s) = \overline{R}_0 - \int_{s_0}^s f(\xi)\,d\xi\,\hat{u}	 
 \]
 
 Isolando il versore tangente e ricordando la sua definizione si ottiene
 \[
\dfrac{dP}{ds}(s)	 = \vtau (s) = \dfrac{1}{T(s)}\left[\overline{R}_0 - \int_{s_0}^s f(\xi)\,d\xi\,\hat{u}\right]
 \]
e integrando tra gli estremi $s_0$ e $s$
\begin{align*}
P(s)	 - P(s_0) =& \int_{s_0}^s \dfrac{1}{T(s)}\left[\overline{R}_0 - \int_{s_0}^s f(\xi)\,d\xi\,\hat{u}\right]\,ds = \\
=& \int_{s_0}^s \dfrac{1}{T(s)}\,\overline{R}_0 - \int_{s_0}^s \dfrac{1}{T(s)}\left[\int_{s_0}^s f(\xi)\,d\xi\right]\,ds\,\hat{u}
\end{align*}

Si evince che, fissato un punto $P(s_0)$, la posizione nello spazio del punto $P(s)$ in riferimento a $P(s_0)$ si può esprimere come combinazione lineare dei vettori $\overline{R}_0$ e $\hat{u}$ costanti nello spazio.

La fune, soggetta esclusivamente a sollecitazioni di direzione costante, giace nel piano generato dai due vettori sopracitati.

\section{Terna di riferimento standard}
Una volta trovata l'equazione indefinita di equilibrio \eqref{eq:equzione_indefinita_equilibrio} è conveniente definire un sistema di riferimento cartesiano $Oxyz$. Nell'ipotesi di forze continue e parallele, sia il versore costante $\hat{u}$ coincidente con il versore $\hat{e}_2$ della terna considerata. 
L'equazione vettoriale \eqref{eq:equzione_indefinita_equilibrio} può essere ricondotta a tre equazioni scalari nelle coordinate $x, y, z$ rispettivamente. Per quanto scritto nel paragrafo precedente, la fune giace nel piano $Oxy$; pertanto, l'equazione lungo la coordinata $z$ può essere trascurata. Il sistema che governa il problema, allora, è
\begin{equation}
	\label{eq:sistema_terna_standard}
	\begin{cases}
		\dfrac{d}{ds}\left(T\,\dfrac{dx}{ds}\right) = 0\\[1.5ex]
		\dfrac{d}{ds}\left(T\,\dfrac{dy}{ds}\right) + f = 0
	\end{cases}
\end{equation}

Integrando la prima delle \eqref{eq:sistema_terna_standard} si ha
\begin{equation}
	\label{eq:tensione_direzione_x}
T\,\dfrac{dx}{ds} = c = cost.	
\end{equation}

Essendo
\[
\overline{T}\cdot\hat{e}_1	= T\,\vtau\cdot \hat{e}_1 = T\,\dfrac{dP}{ds}\cdot\hat{e}_1 = T\,\dfrac{dx}{ds} = c
\]
la componente del vettore tensione $\overline{T}$ lungo la direzione ortogonale al versore $\hat{u}$ risulta, quindi, costante. 

In generale si assume la costante $c$ non negativa. Nell'eventualità che risulti nulla, si desume che 
\[
\dfrac{dx}{ds} = 0	\quad \Longrightarrow \quad x(s) = x_0\quad \forall s\in[s_1, s_2]
\]  
cioè, la fune si posizione su una retta parallela all'asse $y$.

Nel caso più generale in cui $c>0$ ne consegue che sia $T(s)$ che $\frac{dx}{ds}(s)$ siano strettamente positivi per ogni $s\in[s_1, s_2]$. Isolando il termine di variazione di ascissa 
\[
\dfrac{dx}{ds}(s) = \dfrac{c}{T(s)}	> 0
\]

Ne risulta che $x$ è una funzione monotona crescente nella variabile $s\in[s_1, s_2]$ e invertibile, con inversa $s(x)$ di derivata prima
\[
\dfrac{ds}{dx}(x)	= \left.\left(\dfrac{dx}{ds}(s)\right)^{-1}\right|_{s=s(x)} = \left.\dfrac{T(s)}{c}\right|_{s=s(x)}
\]

Si può esprimere la $y$ in funzione della coordinata $x$ come funzione composta
\[
y = y(s) = y(s(x))	
\]

La funzione $y$ è derivabile con derivata prima 
\begin{equation}
	\label{eq:relazione_derivata}
\dfrac{dy}{dx}(x) = \left.\dfrac{dy}{ds}(s)\right|_{s=s(x)}\,\dfrac{ds}{dx}(x) = \left.\dfrac{dy}{ds}(s)\right|_{s=s(x)}\,\left( \dfrac{dx}{ds}(x)\right)^{-1}
\end{equation}

Risolvendo la \eqref{eq:tensione_direzione_x}, isolando il termine di tensione
\[
T = c\,\left(\dfrac{dx}{ds}\right)^{-1}	
\]
e sostituendo quanto appena trovato nella relazione in direzione $y$ del sistema \eqref{eq:sistema_terna_standard} si ricava
\[
\dfrac{d}{ds}\,\left[c\,\left(\dfrac{dx}{ds}\right)^{-1}\,\dfrac{dy}{ds}\right] + f = 0	
\]

Utilizzando la relazione \eqref{eq:relazione_derivata} appena calcolata 
\[
\dfrac{d}{ds}\,\left(c\,\dfrac{dy}{dx}\right) + f = 0	
\]
ed essendo $s = s(x)$ 
\[
\dfrac{d}{dx}\,\left( c\,\dfrac{dy}{dx}\right)\,\dfrac{dx}{ds} + f = 0	
\]
che è equivalente a scrivere
\[
\dfrac{d}{dx}\,\left( c\,\dfrac{dy}{dx}\right)\,\left(\dfrac{ds}{dx}\right)^{-1} + f = 0
\]

Il differenziale dell'ascissa curvilinea è definito come
\[
 ds = \sqrt{dx^2 + dy^2} = \sqrt{dx\left(1 + \left(\dfrac{dy}{dx}\right)^2\right)} = \sqrt{1 + \left(\dfrac{dy}{dx}\right)^2}\,dx
\]
e quindi
\[
\dfrac{ds}{dx} = \sqrt{1 + \left(\dfrac{dy}{dx}\right)^2}	
\]

Sostituendo si ottiene la \emph{equazione cartesiana della funicolare}
\begin{equation}
	\label{eq:cartesiana_funicolare}
	c\,\left[1+ \left(\dfrac{dy}{dx}\right)^2\right]^{-1/2}\,\dfrac{d^2 y}{dx^2} + f = 0
\end{equation}

L'equazione \eqref{eq:cartesiana_funicolare} può essere ridotta a forma normale ponendo $\frac{dy}{dx} = p$ in modo tale che si ricavi il sistema
\[
\begin{cases}
	\dfrac{dy}{dx} = p\\[1.5ex]	
	\dfrac{dp}{dx} = -\dfrac{1}{c}\,f\,\sqrt{1+p^2}\\[1.5ex]
	\dfrac{ds}{dx} = \sqrt{1+p^2}
\end{cases}\quad (y,p,s)\in\mathbb{R}^3
\]

dove $f$ deve essere riscritta in funzione della variabile $x$. Il sistema ammette una sola soluzione imponendo le condizioni iniziali in $x_0\in\mathbb{R}$
\[
\begin{cases}
	\dfrac{dy}{dx} = p\\[1.5ex]	
	\dfrac{dp}{dx} = -\dfrac{1}{c}\,f\,\sqrt{1+p^2}\\[1.5ex]
	\dfrac{ds}{dx} = \sqrt{1+p^2}\\[1.5ex]
	y(x_0) = y_0\\
	p(x_0) = p_0\\
	s(x_0) = s_0
\end{cases}\quad (y_0,p_0,s_0)\in\mathbb{R}^3
\]
mentre il valore della costante $c$ si ottiene una volta noti $p(x_0) = p_0$ e $T(x_0) = T_0>0$
\[
c = T\,\dfrac{dx}{ds} = T\,\left(\dfrac{ds}{dx}\right)^{-1} = \dfrac{T_0}{\sqrt{1+p_0^2}}
\]

Infine, è sufficiente calcolare la tensione nel filo applicando la \eqref{eq:tensione_direzione_x} opportunamente manipolata
\begin{equation}
	\label{eq:calcolo_tensione}
	T(x) = c\left[1+ \left(\dfrac{dy}{dx}\right)^2\right]^{1/2}
\end{equation}

\subsection{Problema a valori al contorno}
Applicando il sistema di riferimento $Oxyz$ nel problema a valori al contorno trattato nel paragrafo \ref{section:condizioni_contorno} ci si imbatte in un problema: essendo noti gli estremi della fune $P(s_1)$ e $P(s_2)$ non è noto il valore costante della tensione $\overline{T}(s_0)$. 
\[
\overline{T}(s_0)	= T(s_0)\,\vtau (s_0)\cdot\hat{e}_1 = c 
\]

Il termine $c$ è una \emph{incognita} da aggiungere al sistema risolvente. Ricordando che, dall'equazione indefinita di equilibrio \eqref{eq:equazione_indefinita_equilibrio_1}
\[
	\dfrac{dT}{ds} = -\dfrac{\overline{f}\cdot\vtau}{\vtau^2}
\]
e che $\overline{f} = f\,\hat{u} = f\,\hat{e}_2$, il sistema finale è
\[
	\begin{cases}
\dfrac{dx}{ds} = \tau_x\\[1.5ex]
\dfrac{dy}{ds} = \tau_y\\[1.5ex]
\dfrac{\tau_x}{ds} = \dfrac{1}{T}\,\dfrac{f\,\tau_y}{\tau_x^2 + \tau_y^2}\,\tau_x\\[1.5ex]
\dfrac{\tau_y}{ds} = \dfrac{1}{T}\left(-f+\dfrac{f\,\tau_y}{\tau_x^2 + \tau_y^2}\,\tau_y\right)\\[1.5ex]
\dfrac{dT}{ds} = -\dfrac{f\,\tau_y}{\tau_x^2 + \tau_y^2}
	\end{cases}
\]

Imposte le 4 condizioni al contorno
\begin{align*}
	&x(s_1) = x_1\in\mathbb{R} \quad &&y(s_1) = y_1\in\mathbb{R}\\
	&x(s_2) = x_2\in\mathbb{R} \quad &&y(s_2) = y_2\in\mathbb{R}
\end{align*}
il sistema è univocamente determinato.
