\begin{figure}[ht]
		\centering
		
		\subfloat[\emph{Tensione in un tratto di filo}\label{fig:tensione_filo}]{
		\begin{tikzpicture}[scale=.8]
		
			
			
			\draw[line width=1pt] (0,0)node [left] {$P(\lambda_1)$} .. controls ++(1.5,3) and ++(.1,-2) .. (10,4) node [right] {$P(\lambda_2)$}
			node[sloped, pos=.3, anchor=south west,
			minimum height=(10.5)*0.3cm,minimum width=(10.5)*.3cm](p){};
			
			\path (p.south west)%
		%	edge[-stealth',blue] node[left] {$\vec{ n}$} (p.north west)
			edge[-stealth',blue, thick] node[pos=1, above] {$\overline{T}(\lambda)$} (p.south east);
			
			\node at (p.south west) [below] {$P(\lambda)$};
			
			
			
		
		
		
		
 						\end{tikzpicture}}
 		
 		\subfloat[\emph{Estensione del postulato a una fune ideale}\label{fig:tensione_fune_ideale}]{
 		
 \begin{tikzpicture}[scale=1]
 
 
 
 
 
 
 
 
 
 \draw plot  [smooth, tension=1] coordinates {(0,0) (1,.3) (3,.6)};
 \begin{scope}[shift={(0,-1)}]
 \draw plot  [smooth, tension=1] coordinates {(0,0) (1,.3) (3,.6)};
 \end{scope}
 
 \draw (0,0) to [out=-20, in=40] (0,-1);
 \draw (0,0) to [out=-145, in=130] (0,-1);
 
 \draw (3,.6) to [out=-20, in=40] (3,-.4);
 \draw [dashed] (3,.6) to [out=-145, in=130] (3,-.4);
 
 \draw [-stealth', blue] (3,.1) --([turn] 8:1) node [above] {\small$\overline{T}(\lambda)$};
 
 \node at (0,0) [left] {$P(\lambda_1)$};
 \node at (3,.6) [above] {$P(\lambda)$};
 
 \begin{scope}[shift={(9,0.35)}, yscale=1, xscale=-1]
 \draw plot  [smooth, tension=1] coordinates {(0,0) (1,.3) (3,.6)};
 \begin{scope}[shift={(0,-1)}]
 \draw plot  [smooth, tension=1] coordinates {(0,0) (1,.3) (3,.6)};
 \end{scope}
 
 \draw [dashed](0,0) to [out=-20, in=40] (0,-1);
 \draw [] (0,0) to [out=-145, in=130] (0,-1);
 
 \draw (3,.6) to [out=-20, in=40] (3,-.4);
 \draw [] (3,.6) to [out=-145, in=130] (3,-.4);
 
 \draw [-stealth', blue] (3,.1) --([turn] -8:1) node [above] {\small$-\overline{T}(\lambda)$};
 
 \node at (0,0) [right] {$P(\lambda_2)$};
 \node at (3,.6) [above] {$P(\lambda)$};
 \end{scope}
 \end{tikzpicture}


}
		
		
		
		
		
		
		
		
		
		\caption{Tensione in un tratto di filo}
		\label{fig:tensione_fune}
		
	\end{figure}
	
	
	