


\begin{figure}
	\centering
	
	\begin{tikzpicture}[scale=.6]
	
	\draw [thick] (0,0) .. controls (5,-.5) .. (10,.5);
	
	
	
	\draw [-latex, thick] (0,-.25) -- ++([turn] 75:-1.5) node [above left] {$-\overline{T}(s)$};
	
	\draw [-latex, thick] (10,.25) -- ([turn] 12:2) node [above right] {$\overline{T}(s + \delta s)$};
	
	
	
	
	\draw [thick](0,-.5) node [below] {$P(s)$} .. controls (5,-1)..(10,0) node [below right] {$P(s+\delta s)$};
	
	\draw[thick] (0,0) .. controls (.2,-.18) and (.2,-.36) .. (0,-.5);
	\draw[thick] (0,0) .. controls (-.2,-.18) and (-.2,-.36) .. (0,-.5);
	
	\draw [thick](10,0.5) .. controls (10.1,.3) and (10.1,.1) .. (10,0);
	\draw [thick, dashed](10,0.5) .. controls (9.9,.3) and (9.9,.1) .. (10,0);
	
	
	
	\begin{scope}[shift={(0,1.3)}]
	\draw [thick]  (0,0) .. controls (5,-.5) .. (10,.5);
	\draw [forcedist=.75cm]  (0,0) .. controls (5,-.5) ..(10,.5);
	\node  at (5,0.2) {$\overline{f}(\xi)$};
	\end{scope}
	
	
	\end{tikzpicture}
	
	\caption{Forze esterne agenti su un tratto di filo}
	\label{fig:forze_filo}
	
\end{figure}


